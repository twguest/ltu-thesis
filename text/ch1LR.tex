


\section{Properties of X-rays}


\subsection{First Principles: Maxwell and Helmholtz Equations}
We begin our disucssion of X-rays with the free-space Maxwell equations:

\begin{align}
    \divergence \efield \spacetime &=0 \label{eq: maxwell 1} \\
     \divergence \bfield \spacetime &=0 \label{eq: maxwell 2} \\
     \curl \efield\spacetime +  \tder \bfield\spacetime &=0 \label{eq: maxwell 3} \\
     \curl \bfield\spacetime - \epsilon_0\mu_0 \tder \efield\spacetime &=0 \label{eq: maxwell 4} 
\end{align}
where \efield\ and \bfield\  are the electric field and magnetic induction vectors \footnote{where the bold format, unless stated otherwise, will be used throughout this text to denote vector notation}, $\epsilon_0$ and $\mu_0$ are the electrical and magnetic permitivity of free-space, and $\nabla$ and $\nabla \times$ denote the gradient and curl operators of the three-dimensional coordinate system: $x,y,z \in \mathbb{R}$. Using the vector identity [\eqn{\ref{eq: vector identity 1}}] and taking the curl of \eqn{\ref{eq: maxwell 3}} we get
 
\begin{align}
    \grad\left[\divergence\efield\spacetime\right]-\grad^2\efield\spacetime&+\curl\tder\bfield\spacetime = 0 \\
    \tf \grad^2\efield\spacetime &= \ttder\epsilon_0\mu_0\efield\spacetime
\end{align}
which can be used to obtain the d'Alembert wave equation using the fact that $\epsilon_0\mu_0 = c^{-2}$:
\begin{equation}\label{eq: d'Alembert equation}
    \left(\frac{1}{c^2}\ttder-\grad^2\right)\efield\spacetime = 0,
\end{equation}
which can be equivalently obtained for the magnetic induction vector:
\begin{equation}\label{eq: d'Alembert equation B}
    \left(\frac{1}{c^2}\ttder-\grad^2\right)\bfield\spacetime = 0,
\end{equation}
which illustrates that each of the three spatial components of the free-space electric field and the free-space magnetic induction are uncoupled from each other. \\

Finally, we note that either of these vector fields can be replaced by a single, complex scalar field function $E$ describing an electromagnetic disturbance in space-time:
\begin{align}\label{eq: d'Alembert equation scalar}
    \left(\frac{1}{c^2}\ttder-\grad^2\right)E\spacetime = 0,
\end{align}
which has the observable optical intensity $I$:

\begin{align}
    I\spacetime = \left|E\spacetime\right|^2.
\end{align}


\subsection{Propagation of X-rays in a Free-Space}
Any electromagnetic field that obeys the d'Alembert equation (\eqn{\ref{eq: d'Alembert equation scalar}}) can be decomposed into a superposition of its monochromatic components via the Fourier transform:
\begin{equation}\label{eq: monochromatic signal}
    E\spacetime = \frac{1}{\sqrt{2\pi}}\int_0^\infty E_\om \spatial \carrier d\om
\end{equation}
where each $E_\om$ denotes the spatial wavefield\footnote{Here we choose to distinguish between a wavefield as an electromagnetic disturbance defined in 3 spatial and 3 phase-space dimensions, as opposed to a wavefront, which we define to be a surface of constant phase. We restrict any use of the term wavefunction to describe probabilistic fields.} for a single temporal frequency $\om$. We label the RHS of \eqn{\ref{eq: monochromatic signal}} as the monochromatic signal form of the electromagnetic disturbance E. Substituting \eqn{\ref{eq: monochromatic signal}} into \eqn{\ref{eq: d'Alembert equation scalar}} and defining the wavevector $k: k = \frac{\om}{c}$ we obtain the Helmholtz equation for each of the components of the decomposed polychromatic field:
\begin{equation}
    \left(\grad^2 + k^2\right)E_\om\spatial = 0.
\end{equation}


\subsection{Angular Spectrum Representation}
With an understanding of the spectral decomposition of a wavefield, we move to define an operator which models the evolution of any suitably defined wavefield in an arbitrary parallel plane, \ie we pursue $E(x,y,z=\dz)$ from $E(x,y,z=0)$.\\

Construct a Cartesian coordinate system with a monochromatic plane-wave defined in the plane of origin at $z=0$. Maintain that the wave is forward propagating: there is no points in the plane of origin where the flow of optical energy is right-to-left. Let $E_{\bar{\om}}$ be the component of the monochromatic signal corresponding to the mean spectral frequency component of $E$ and consider the wave-vector $k$ to be the sum of separable spatial components: \kx, \ky and \kz. Hence, the longitudinal wavevector of any solution to the Helmholtz equation can subsequently be defined: 
\begin{equation}
    \kz^2 = \sqrt{k^2-\kx^2-\ky^2}.
\end{equation}

which enables us to write the plane-wave disturbance at origin:
\begin{equation}\label{eq: plane wave}
     E_\om\spatial = \exp\left[i\left(\kx x + \ky y + \kz z\right)\right]
\end{equation}
as the product of its transverse and longitudinal elements:

\begin{align}\label{eq: plane wave equation}
    E_\om(x,y,z=0) &= \exp\left[iz\sqrt{k^2-\kx^2-\ky^2}\right]\\
    &= \exp\left[i\left(\kx x + \ky y\right)\right]\exp\left[iz\sqrt{k^2-\kx^2-\ky^2}\right].
\end{align}

From \eqn{\ref{eq: plane wave equation}} we see that:

\begin{equation}
    E_\om(x,y,z=0) = \exp\left[i\left(\kx x + \ky y \right)\right]
\end{equation}

\noindent which means that the evolution of the unpropagated field $E(x,y,z=0)$ is obtained by multiplication of the initial wavefield by the free space operator $\exp\left[iz\sqrt{k^2-\kx^2-\ky^2}\right]$.\\

We can represent any $E_\om$ as the two-dimensional Fourier integral describing the linear contribution of its plane-wave spatial components:

\begin{equation}\label{eq: plane wave decomposition}
    E_\om(x,y,z = 0) = \frac{1}{2\pi}\iint E_\om(\kx,\ky,z=0)exp\left[i(\kx x + \ky y)\right]d\kx d\ky
\end{equation}

where each spatial component in the integral on the RHS of \eqn{}



\begin{align}
        \notag E_\om(x,y,z = \dz) &= \frac{1}{2\pi}\iint  E_\om(\kx,\ky,z=0)exp
        \left[i(\kx x + \ky y)\right]\\
        &\times \exp\left[iz\sqrt{k^2-\kx^2-\ky^2}\right]
        d\kx d\ky
\end{align}
From which we define the free-space diffraction operator, $\mathcal{D}_{\dz}$:

\begin{equation}\label{eq: fresnel diffraction operation}
    E_\om(x,y,z=\dz) = \mathcal{D}_{\dz}E_\om(x,y,z=0)
\end{equation}

\begin{equation}\label{eq: fresnel diffraction operator}
    \mathcal{D}_{\dz} = \ifourier \exp\left[iz\sqrt{k^2-\kx^2-\ky^2}\right] \fourier 
\end{equation}



where \fourier \ and $\ifourier$ \ are the Fourier and inverse Fourier transforms with respect to transverse coordinates.

With this formulation, we present an algorithm for propagating any polychromatic scalar disturbance to an arbitrary parallel plane:
\begin{enumerate}\label{algorithm: fresnel diffraction}
    \item Take the angular decomposition of the polychromatic field, $E$ into its monochromatic components $E_\om$ via Equation \ref{eq: monochromatic signal}.
    \item Decompose each monochromatic component of the monochromatic signal into its plane-wave spatial components via Equation \ref{eq: plane wave decomposition}.
    \item Multiply each of the plane wave components by the free-space Fresnel diffraction operator given in Equation \ref{eq: fresnel diffraction operator}.
    \item Recompose each of the propagated monochromatic components as the sum of its spatial spatial frequencies by taking the inverse Fourier transform of the R.H.S of \ref{eq: plane wave decomposition}.
    \item Recompose the propagated, polychromatic electric field from the sum of its monochromatic components by taking the inverse Fourier transform of the R.H.S of \ref{eq: monochromatic signal}.
\end{enumerate}

\subsection{Fresnel Diffracton}\label{ss: Fresnel diffraction}
Collimated optical fields are considered to be paraxial when all non-negligible plane wave components make a small angle with respect to the optical axis. In this case, the longitudinal wavevector can be expressed via the binomial approximation:

\begin{equation}\label{frensel binomial}
\sqrt{k^2-\kx^2-\ky^2}\approx k-\frac{\kx\2+\ky\2}{2k}
\end{equation}
allowing the free-space propagator \D to be re-written for the propagation of paraxial wavefields:

\begin{equation}\label{eq: Fresnel Operator}
\DF \equiv \exp\left(i k \dz\right)\ifourier\exp\left[\frac{i\dz \left(\kx\2+\ky2\right)}{2k}\right]
\end{equation}


\begin{align}\label{eq: Frensel Diffraction}
\notag E_\om(x,y,z = \dz) &\equiv \DF E_\om(x,y,z=0) \\
&=  \exp\left(ik\dz\right)\ifourier\exp\left[\frac{i\dz \left(\kx\2+\ky\2\right)}{2k}\right]\fourier E_\om(x,y,z = 0)
\end{align}
 
where the paraxial Fresnel free-space diffraction operator is labelled \DF

\subsection{Sampling Requirements of the Fresnel Propagator}
The issue of adequately sampling a wavefield pertains to the loss of information due to the digitisation of a continuous signal - undersampling manifests as non-physical signal artefacts due to loss of phase information of high spatial frequencies in the discretisation of fields in numerical applications, as well as physical applications (\ie the discretisation of a physical field due to the finite resolution of an optical intensity detector). For a paraxial field, as in \ref{ss: Fresnel diffraction}, discrete representations of the field dictate finite step sizes in the \fourier and \ifourier. Considering first the computational case \footnote{where the available parameter space is less restrictive} implementation of a discrete Fourier and inverse Fourier samples on each of the transverse dimensions must sample the continuous space at an interval smaller than or equal to the Nyquist limit, $\boldsymbol{N}$:

\begin{equation}\label{eq: Nyquist Limit}
    \boldsymbol{N} \leq \pi k_{max}^{-1}= \pi\sigma_{max}
\end{equation}

where $k_{max}$ and $\sigma_{max}$ are the extent of the wavefield in frequency and real-space. This denotes that the approximation in \eqn{\ref{eq: Frensel Diffraction}} if the phase-gradient of adjacent spatial frequencies is less than $\pi$. An expression for the satisfactory spatial step-size of the discrete field can be defined considering the maximum phase of a paraxial field propagated a distance $\dz$ in free-space \footnote{which occurs at the boundary of the continuous wavefield defined in real-space}:

\begin{equation}\label{eq: Phase of Fresnel Propagator}
    \phi^{(F)}_{max} = \frac{k}{2\dz \sigma_{max}}, \quad
    \tf \frac{\partial \phi^{(F)}_{max}}{\partial k_{max}} = \pi \sigma_{max}.
\end{equation}
For simplicity, consider the wavefield to be axiosymmetric so that it may be defined on the interval $\left[\frac{-\sigma_{max}}{2}, \frac{\sigma_{max}}{2}\right]$ which has a corresponding Fourier space domain $\left[\frac{N}{-2\sigma_{max}}, \frac{N}{2\sigma_{max}}\right]$ where $N$ is the number of discrete points over which the continuous wavefield has been sampled. Defining a sampling interval $dN = \frac{\sigma_{max}}{N}$ and making use of the fact $k = \frac{2\pi}{\lambda}$, as well as \eqn{\ref{eq: Nyquist Limit}} and \eqn{\ref{eq: Phase of Fresnel Propagator}} we can define the minimum sampling frequency of a paraxial field propagated through free space:

\begin{equation}\label{eq: sampling requirements}
    dN \leq \frac{\dz \lambda}{\sigma_{max}}.
\end{equation}
In experiment, $dN$ corresponds to the minimum spatial resolution of an optical intensity detector to avoid artefacts in representing an optical wavefield. We note that in many experimental cases this value is fixed and the issue of sampling is solved by shifting the detector downstream. 

\subsection{The Huygens-Fresnel Convolution Propagator}
\eqn{\ref{eq: sampling requirements}} imposes severe restrictions on the complexity of fields that can be represented computationally as the size of any $E_\om(x,y)$ scales with $N^2$. Making use of the Fourier convolution theorem (link to appendix) by noting that the convolution of any two well behaved functions $f(x,y)$ and $g(x,y)$ can be written in terms of \fourier and \ifourier:

\begin{equation}\label{eq: fourier convolution statement}
f(x,y) \circledast g(x,y) = 2\pi \ifourier\left\{\fourier \left[ f(x,y) \right]\fourier\left[g(x,y) \right] \right\}
\end{equation}
allowing us to write \ref{eq: Frensel Diffraction}

\begin{align}\label{eq: Huygens Fresnel Diffraction}
\notag E_\om(x,y,z = \dz) &= 
2\pi\ifourier \{ \fourier\ifourier \left[\frac{1}{2\pi}\exp(ik \dz)\exp\left(\frac{-ik(\kx\2+\ky\2)}{2k}\right)\right] \\
&\times\fourier \left[E_\om(x,y,z = 0)\right]\} \\
&\equiv E_\om(x,y,z=0)\circledast\DHF(x,y,z=\dz)
\end{align}
where we define \DHF to be the Huygens-Fresnel free-space propagation operator which formulates the propagated wavefield as the sum of electromagnetic disturbances propagated from each sampled point in the unpropagated plane as spherical waves (cite): 
%%%%% note x_0 should be changed to x' but requires a latex fix re: double superscripts
\begin{align}\label{eq: Huygens Fresnel convolution}
\notag E_\om(x,y,z=\dz) &\approx \frac{-ik\exp(ik\dz)}{2\pi\dz}\iint E_\om(x_0,y_0,z = 0) \\
&\times\exp\{\frac{ik}{2\dz}\left[(x\2-x_0\2)+(y\2-y_0\2) \right]\}dx_0dy_0
\end{align}
The absence of  a Fourier transform on the RHS of \eqn{\ref{eq: Huygens Fresnel convolution}} provides a convenient method of circumventing sampling requirements in numerical propagation of free-space diffraction. 

\subsection{Fraunhofer Diffraction}
Finally, we consider the limiting case where the propagation distance is far greater than the characteristic length scales of the wavefield. Factorising \eqn{\ref{eq: Huygens Fresnel convolution}}:

\begin{align}\label{eq: factorised huygens fresnel}
\notag E_\om(x,y,z=\dz) &\approx \frac{-ik\exp(ik\dz)}{2\pi\dz}
\exp\left[\frac{ik(x\2+y\2)}{2\dz}\right]
\iint E_\om(x_0,y_0,z = 0) \\
&\times\exp\{\frac{ik}{2\dz}\left[(x\2-x_0\2)+(y\2-y_0\2) \right]\}dx_0dy_0
\end{align}




\section{X-ray Imaging}
X-ray imaging is at the core of this. X-rays are paticularly advantadges due to this, and are implemented in myriad techniques cite cite cite. In this section we describe the challenges associated with contemperary image techniques without a view of XFel specific techniques which we reserve for this section... 


\subsection{Interaction of Light with Matter}
The interaction of light with matter is at the center of x-ray imaging. The structural information incoded in the wavefield during the coupling of the light and matter fields enables interpretation of the three-dimensional structure of the material. Here we pursue a description of the evolution of electromagnetic fields in the presence of a scattering medium, beginning with the generalised Maxwell's equations:

\begin{align}
    \divergence \boldsymbol{D} \spacetime &= \rho\spacetime \label{eq: maxwell 5} \\
     \curl \boldsymbol{H}\spacetime - \tder \boldsymbol{D}\spacetime &= \boldsymbol{J}\spacetime \label{eq: maxwell 6} 
\end{align}
\noindent where we have introduced the electric displacement vector, $\boldsymbol{D}$, and the magnetic field $\boldsymbol{H}$, charge density $\rho$ and current density $\boldsymbol{J}$. We note that \eqn{\ref{eq: maxwell 2}} and \eqn{\ref{eq: maxwell 3}} hold as in free-space. \eqn{\ref{eq: maxwell 5}} is Gauss' law and states the proprtionality of the charged enclosed within a surface and the lectric displacement flux through the surface. \eqn{\ref{eq: maxwell 6}} is Maxwell's modification of Gauss' law and denotes that magnetic fields can be induced by both electric current or a changing displacement field.\\

In general, the displacement and induction vectors are functions of \efield and \bfield \footnote{in ferroelectric and ferromagnetic materials, $\boldsymbol{D}$ and $\boldsymbol{B}$ depend on the history of the material}. Restricting ourselves to the case of linear materials, let $\boldsymbol{D} = \epsilon\efield$ and $\bfield = \mu\boldsymbol{H}$ where $\epsilon$ and $\mu$ denote the electrical and magnetic permetivity of the materials. Hence, we can re-write Maxwell's equations in terms of \efield and \bfield:
\begin{align}
    \divergence \left[\epsilon\spacetime\efield\spacetime\right] &= \rho\spacetime \label{eq: maxwell 7} \\
     \curl \left[\frac{\bfield\spacetime}{\mu\spacetime}\right] - \tder\left[\epsilon\spacetime\efield\spacetime\right] &= \boldsymbol{J}\spacetime \label{eq: maxwell 8} 
\end{align}

%%%%%%% via masters
The interaction of any material, M, with BEUV and Soft X-ray (SXR) radiation is characterised by the energy dependent complex refractive index of the material, $n$:

\begin{equation}\label{complex_refractive_index}
    n_M^{(E)}(x,y,z) = 1 - \delta_M^{(E)}(x,y,z) - i\beta_M^{(E)}(x,y,z).
\end{equation}
The real and imaginary components of $n$ describe the phase and amplitude shift of a complex electric field propagated through the material and are defined by \cite{paganin_coherent_2013}:

\begin{align}\label{Eqn: Scattering Factors}
\delta &= \frac{r_e \lambda^2 n_e}{2\pi}, \text{and}\\
\beta &=  \frac{\mu^{(E)}_M}{2k},
\end{align}
where $r_e$ is the classical electron radius, $n_e$ the electron density in the material and $\upmu$ the material's absorption coefficient, which is defined by the Beer-Lambert law:
\begin{equation}\label{absorption_coeff}
    I(x) = I_0 exp\left[-\mu_M^{E} x\right].
\end{equation}


An incident wavefield $\psi(x,y,z = 0)$ entering a material, M, with a spatially dependent thickness $\tau(x,y): 0 < \tau \leq z_0$, takes the form in the exit plane:
\begin{equation}
\begin{aligned}
    \psi(x,y,z = z_0) = &exp\left\{-ik\int_{z = 0}^{z = z_0} \left[\delta_M^{(E)}(x,y,z) - i\beta_M^{(E)}(x,y,z)\right]dz\right\} \\ &\times \psi(x,y,z = 0),
\end{aligned}
\end{equation}
where the phase and amplitude perturbations imparted on the the incident wavefield by the material are given:
\begin{equation}
    \Delta\psi(x,y) = k \delta^{(E)}_M \tau(x,y),
\end{equation}
and
\begin{equation}
    I(x,y,z = z_0) = exp \left[-2k \int_{z = 0}^{z = z_0} \beta_M^{(E)}(x,y,z) dz\right]I(x,y,z = 0).
\end{equation}
%%%%%%% 

\subsection{Optical Coherence}
Until now we have restricted ourselves to the case of idealised monochromatic or polychromatic plane-wave sources where the ensemble of electromagnetic fields, $\Psi$, contains only a single disturbance \footnote{Here $\Psi$ is used to represent the ensemble set of electromagnetic fields at some point in time, where we have intentionally evoked notions similar to the continuous quantum mechanical wavefunctions describing the probabilistic set of states of the EM field.}. In reality, a thorough description of X-ray wavefields must admit that the quantum and thermal processes that govern the emission of X-ray fields are inherently probabilistic \footnote{this is not to mention the probabilistic nature of the interaction of fields with matter} and the resulting X-ray wavefields are stochastic - we label these fields to be partially coherent and note that coherence is achieved in the limit that the density of the random fluctuations (both spatial and temporal) approaches 0, and incoherent as this density approaches infinity (is this even valid?). This discussion of coherence is pertinent to the case of XFEL wavefields generated by the quasi-spontaneous emmission of thermal distributions of relativistic electrons, where the field ensemble measured at the exit of the undulator is populated by couple radiation fields from large numbers of individual emmitters with phases that fluctuate randomly in spacetime.\\

To describe the degree of coherence of an ensemble of electromagnetic field disturbances, we contrive the following illustrative case: Consider a set of emmitters occupying a small volume and radiating quasi-monochromatic light at a mean wavelength $\bar{\lambda}$. For now, we consider the radiation of each emmitter to be statistically stationary in the wide-sense: the correlation between any two temporal realisations of that field depends only on the time-lag between the realisations \footnote{nonetheless, it is significant that we hold that this cannot always be assumed for XFEL wavefields.} Suppose the emmitting volume is stationary and located upstream of an thin opaque screen with a pair of pinhole perforations at locations $\boldsymbol{r}_{\perp,1} = (x_1,y_1)$ and $\boldsymbol{r}_{\perp,2} = (x_2,y_2)$ in the transverse plane. The position vector describing the relative displacement of the pinholes is therefore given: $\Delta\boldsymbol{r}_\perp = \sqrt{(x_2-x_1)^2+(y_2-y_1)^2}$. We consider a single-pixel (bucket) detector located some distance \dz downstream and let $\rpara_{,1}$ and $\rpara_{,2}$ be the paths between each pinhole and the detector. The resulting disturbance measured at the detector is effectively a two-point correlation measurement of the wavefield $E_\Phi$(x,y,z=0) in the plane of incidence with the screen.\\

%The statement of quasis-monochromaticity is equivalent to saying that the non-zero components of the monochromatic signal lie within a small bandwidth on the interval $\bom - \half\Delta\om \leq \om \leq \bom + \half\Delta\om$
%\begin{equation}
%    E(x,y,t) = \frac{1}{\sqrt{2\pi}}\int_{\bom-\half \Delta\om}^{\bom+\half\Delta\om}E_\om\carrier d\om 
%\end{equation}


%\begin{equation}\label{eq: quasi-monochromatic signal}
%    E(x,y,t) = \frac{1}{\sqrt{2\pi}}\int_{\bom-\half\Delta\om}^{\bom+\half\Delta\om}E_\om\carrier d\om 
%\end{equation}
%restricting the limits of integration to enclose only the non-negligible components of the monochromatic signal: $\om = \bom+\delta\om$, the \eqn{\ref{eq: quasi-monochromatic signal}} becomes:

%\begin{equation}\label{eq: quasi-monochromatic signal}
%    E(x,y,t) = \frac{1}{\sqrt{2\pi}}\int_{\bom-\half\Delta\om}^{\bom+\half\Delta\om}E_\om\carrier d\om 
%\end{equation}
%%%
The disturbance measured at the detector corresponds to the time-averaged interference pattern of disturbances $E_1(x_1,y_1,z=0,t)$ and $E_2(x_2,y_2,z=0,t)$ emanating from each pinhole. If the enemble field $E_\Phi$ is a plane-wave the nature of the superposition of the fields is determined by the path difference $\Delta\rpara = \|\rpara_{,2}-\rpara_{,1}\|$ which will be constructive or destructive when $\Delta\rpara = 0$ or 1, respectively. The phase retardations are a consequeunce of the finite speed of light in a vacuum, c. Knowing this, we can write the measured wavefield, $E_3(x_3, y_3, z=\dz,t)$:

\begin{align}\label{eq: superimposed field}
\notag E_3(x_3, y_3, z=\dz,t) &= K_1E(x_3,y_3,z=\dz, t-\frac{\rpara_{,1}}{c}) \\ &+ K_2E(x_3,y_3,z=\dz, t-\frac{\rpara_{,2}}{c}),
\end{align}

\noindent where $K_1$ and $K_2$ are complex numbers encapsulating information about the diffraction from the pinholes, scattering angles etc. The intensity of the recorded disturbance is given by the ensemble average of the square-modulus of \eqn{\ref{eq: superimposed field}}:

\begin{align}\label{eq: intensity interference}
    \notag I(x_3,y_3,z=\dz) &= \left\langle \left\| K_1E(x_3,y_3,z=\dz, t-\frac{\rpara_{,1}}{c}) + K_2E(x_3,y_3,z=\dz, t-\frac{\rpara_{,2}}{c}) \right\| \right\rangle \\
    \notag &= I_1(x_1,y_1,z=0) + I_2(x_2,y_2,z=0) \\ &+ 2\|K_1K_2^*\|Re\left\langle E_1(x_1,y_1,z=0, t-\Delta t)E*(x_2,y_2,z=0, t)\right\rangle.
\end{align}

\noindent where $K_2^*$ denotes the complex conjugate of $K_2$ and the angular brackets are used to denote the ensemble average over time. In the absence of either one of the pinholes, it is evident that \eqn{\ref{eq: intensity interference}} is reduced to the intensity pattern of the remaining pinhole. Note that we have written the time difference $\Delta t = \frac{\rpara_{,2}-\rpara_{,1}}{c}$. Importantly, the recorded intensity is defined by the interference term on the RHS of \eqn{\ref{eq: intensity interference}}, which is the correlation of the two-fields at the pinholes, where the phase-shift due to an optical path difference has been imposed. We define this to be the mutual coherence function, $G$, which is a function of the location of each pinhole in the transverse plane:

\begin{equation}\label{eq: mutual coherence function}
G(x_1, y_1, z_1; x_2, y_2, z_2; \Delta t) = \left\langle E_1(x_1,y_1,z_1, t-\Delta t)E*(x_2,y_2,z_2, t)\right\rangle,
\end{equation}
\noindent where we have now relaxed the condition that each of the pinholes exist in the same transverse plane. The mutual coherence function (\eqn{\ref{eq: mutual coherence function}} predicts the visibility of fringes in a diffraction experiement (\ie Young's Double Slit experiment) and is therefore a measure of the correlation of any two points of a field, or ensemble of, that can be directly measured. Normalising the mutual coherence function gives the complex degree of coherence, $g$:

\begin{align}\label{eq: complex degree of coherence}
\notag g(x_1, y_1, z_1; x_2, y_2, z_2; \Delta t) &= \frac{G(x_1, y_1, z_1; x_2, y_2, z_2; \Delta t)}{\sqrt{G(x_1, y_1, z_1, t = 0)G(x_2, y_2, z_2, t = 0)}} \\
& = \frac{G(x_1, y_1, z_1; x_2, y_2, z_2; \Delta t)}{\sqrt{I_1(x_1,y_1,z_1,t=0) I_2(x_2,y_2,z_2,t=0)}}
\end{align}

\noindent which describes the coherence of the source on the interval $g \in [0,1]$, where $g = 0$ and $g = 1$ denote incoherence and coherence respectively.\\

Finally, we seek to generalise \eqn{\ref{eq: complex degree of coherence}} by noting that such a correlation measurement is suitable for any wavefield described in real-space and can be liberated from the need for a suitable interferometric experiment. For an statistically stationary x-ray wavefield in a volume of free-space, $\Omega$, the complex degree of coherence of the field can be described for any set of coordinates $\rb_1 , \rb_2 \in \Omega$ without the need for measurement \footnote{Although we are note restricted to the case of a pair of points and in some circumstances are only capable of observing wavefield coherence structures (\ie phase-vortices) using higher order correlation functions}.

\subsection{Partial Coherent Imaging}
In this section we describe the role of partial coherence in defining the output of linear imaging systems, adopting the space-frequency description of partial coherence \cite{wolf_new_1982, mandel_optical_1995}
which is broadly applicable to a discussion of the properties of XFEL pulse ensembles in Chapter \ref{c: Pointing Instability at the SPB-SFX Instrument} and phase contrast imaging and phase retrieval mechanisms Chapter \ref{Phase-Retrieval Methods for MHz XFELs}.\\

Begin with a paraxial complex scalar partially coherent wavefield, $\psi_\omega$ described by an ensemble of monochromatic components of frequency \om:

\begin{align}\label{eq: space-frequency representation}
\psi_\om &= \left\{ E_\omega^{{j}}(x,y) c^{(j)} \right\}\\
\Psi &= \int \psi_\om(x,y) d\om.
\end{align}
\noindent $\Psi$ is the sum of each of the monochromatic components (\ie those obtained from a polychromatic wavefield via \eqn{\ref{eq: monochromatic signal}}) each of which is comprised of the modal contributions of the source at frequency, \om. To maintain generality, we allow $\Psi$ to be the ensemble of all disturbances emmitted from some stochastic source, where the relative weight of the $j^{th}$ contributing emission element to the ensemble is defined to be $c^{(j)}$. For now, we do not restrict $c^{(j)}$ to have any dependence on spectral frequency.\\

\newcommand{\weight}{\ensuremath{c^{(j)}}}
\newcommand{\supj}{\ensuremath{^{(j)}}}

\newcommand{\CSD}{\ensuremath{W_\omega}}

With the purpose of classifying the response of an imaging system to a partially coherent field, we define the cross-spectral density of the each monochromatic ensemble, \CSD:

\begin{equation}\label{eq: cross-spectral density}
    \CSD(x_1, y_1, x_2, y_2) \equiv \langle E_\om(x_1,y_1)\supj E_\om(x_2,y_2)\supj  \rangle
\end{equation}

\noindent where angular brackets denote the ensemble average over all $J$ realisations of ensemble. The associated spectral density is the ensemble averaged absolute square of the field associated with each angular frequency:

\begin{equation}\label{eq: spectral density}
    S_\om \equiv \CSD(x_1, x_2, y_1, y_2) = \langle \|E_\om\supj(x,y)\weight \| \rangle,
\end{equation}
\noindent which is the weighted average over all intensities in the set of monochromatic disturbances, giving the intensity distribution of the field to be $I$:
\begin{equation}\label{eq: intensity from SD}
    I(x,y) = \int S_\om (x,y) d\om.
\end{equation}

\eqn{\ref{eq: cross-spectral density}} is a descriptor of the coherence of a partially coherent field whose statistics are stationary (\ie the correlation between each realisation of field is dependent on the time-lag between each realisation) and ergodic - which we have implicitly denoted by equating time and ensemble averages. We note that if Gaussian statistics can be assumed, all higher orders of the coherence function can be retrieved from \CSD. To explore the implications of partial coherence in an imaging context, we note that the cross-spectral density is the Fourier transform of the mutual coherence function defined in \eqn{\ref{eq: mutual coherence function}}.\\

Consider $\Psi$ to be a partially coherent field at the entry-plane of a linear imaging system with a transmission function $\mathcal{T}_N = \mathcal{T}_1\mathcal{T}_2\mathcal{T}_3...\mathcal{T}_n$ for $n \in N$. Let $\mathcal{T}_{n-1}$ be the object of interest and $\mathcal{T}_n$ be the free-space propagation form the exit-surface of the object to the detector. An example of such a field is presented in \cite{pelliccia_coherence_2012}. Assume that the object is static, non-magnetic and elastically scattering, and that the spatial features of the object vary on length scales far greater than the radiation wavelength. The ensemble of fields at the detector is given as the coherent sum of all monochromatic components, which we define:

\begin{equation}\label{eq: exit plane pc field}
    \psi_\om(x,y,z = \dz) = \left\{E_\om\supj(x,y,z = 0) \weight \mathcal{T}_N(x,y) \right\}.
\end{equation}
\noindent For brevity, we have ignored the fact that the complex transmission function of the object is likely to be frequency dependent: $\mathcal{T}(x,y) = \int \mathcal{T}_\om(x,y) d\om$.\\

Immediately, such a description of the partially coherent wavefield permits the retrieval of a suite of coherence metrics (\ie cross-spectral density, spectral density, Wigner function, vorticity) \cite{paganin_tutorials_2019}. Within the context of contemporary X-ray imaging techniques however, we are instead concerned with the role of the coherence function of the wavefield in defining the detected intensity. While the challenges of partially coherent imaging are explored further in Chapter \ref{The SPB-SFX Mode}, we note here that the implication of \eqn{\ref{eq: intensity from SD}} and \eqn{\ref{eq: exit plane pc field}} is that the information of the coherence structure of a field is encoded in its intensity distribution. Without knowledge of the phase of the source, the coherent intensity component of any image cannot be decoupled from the coherence function of the field ensemble \cite{clark_high-resolution_2012}. From an experimental viewpoint, $I(x,y)$ is subject to averaging of each of the monochromatic components of the field over the exposure time of the detector, and the spatial averaging over each detection element.  The challenge of partially coherent imaging is therefore largely related to the generation of structure in the coherence function of the diffracted field that is unresolved in the digitalised image. Without omission of the pervasive role of image speckle in partially coherent imaging \cite{nugent_coherence_2003, vartanyants_origins_2003} we maintain that the implications of partial coherence in an imaging context pertain not to the loss-of-information, but the construction of information on length scales unresolved by the optical system \footnote{A particularly illuminating quotes is given within the context of optical fields via \cite{vartanyants_origins_2003}: "It is important to
note that the ‘decoherence’ effect of optics is not a degradation of the inherent source coherence, but instead the
creation of an entirely new component to the coherence
function with a dramatically reduced coherence length."}. The ramifications of this will be further discussed in Chapter \ref{Phase-Retrieval Methods for MHz XFELs}.



\subsection{XFEL Radiation}
\subsubsection{Fundamentals of Synchrotron Radiation}
\subsubsection{The SASE Process}