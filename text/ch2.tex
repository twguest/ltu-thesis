




\section{Introduction}
The small size and scattering cross-section of biomaterials necessitates the requirement for high brightness X-ray sources. While the SASE method is capable of producing high-flux X-ray wavefields with high spatial coherence, microbunching instabilities lead to incoherent temporal beam distributions. 

\subsection{Challenges in Modern Imaging using Hard X-rays}
The resolution of contemporary imaging techniques is limited both by the amplification of image aberrations to produce holographic speckle in coherent imaging systems \cite{goodman_fundamental_1976,deng_coherence_2017}, and loss-of-information due to feature blurring in partially coherent imaging systems \cite{nugent_coherence_2003}. In general, diffraction based imaging techniques, \ie Coherent Diffractive Imaging (CDI) \cite{krenkel_phase-contrast_2015, zabler_optimization_2005, miao_extending_1999} and Holographic/Fresnel CDI \cite{snigirev_possibilities_1995, cloetens_holotomography_1999, williams_fresnel_2006, nugent_diffraction_2005}, require a high degree of partial coherence. The implication of partial coherence is non-trivial and has been extensively discussed within the context of image reconstruction and phase contrast imaging techniques \cite{gureyev_spatial_2016, vartanyants_origins_2003, hagemann_coherence-resolution_2018} and is intrinsically linked to definitions of source fluence requirements \cite{hagemann_fluenceresolution_2017,nave_achievable_2020,paganin_spatial_2019}. 
 

\subsection{The SPB-SFX Instrument}




\section{Method}


\subsection{The Coherent Source Model}

\subsubsection{Description and Validation of the Analytical Source Model}
Here we describe the construction and validation of a coherent source model that enables a study of the influence of the optical transport system in the absence of the implications of partial coherence. We note that this source provides a description of the shape and size of the SPB-SFX source using analytical data, but is a poor descriptor of the stochastic nature of XFEL pulses.\\ 

\noindent We begin with a Gaussian source defined in $\mathbb{R}^3$ at the undulator exit:

\begin{align}\label{eq: gaussian pulse}
\notag E(x,y,z = z, t) &= \frac{4E_0}{\tau\sqrt{2\pi}} \frac{2\sigma_0}{2\sigma}\exp\left[\frac{(x^2+y^2)}{2\sigma^2}\right]\exp\left[\frac{k\sqrt{(x^2+y^2)}}{2R(z)}\right]\\
&\times \exp\left[ikz\right]\exp\left[\left(\frac{-2t}{\tau}\right)\2\right]
\end{align}

\noindent where $2\sigma_0$ is the beam width at its waist, which we define to be some distance $z > 0$ upstream of the undulator exit \footnote{where we recognise that the issue of true-source point is rarely solved for XFEL sources}, $2\sigma$ is the beam radius a distance, z, downstream of the beam waist, $E_0 = E(x=0,y=0,z = 0, t = 0)$  and $\tau$ is the duration of the Gaussian pulse. Recognising that the radius of curvature of the beam, $R(z)$ can be written in terms of the beam divergence, $\theta$:

\begin{equation}\label{eq: gaussian radius of curvature}
    R(z) = z\left[1+\frac{\lambda}{\pi z\theta^2}\right].
\end{equation}

With the goal of building a Gaussian beam model with properties defined by analytical models, we aim to describe the beam in terms of measurable quantities, \ie we aim to write \eqn{\ref{eq: gaussian pulse}} in terms of $2\sigma$ and $\theta$. Moving forward, we note that

\begin{align}
    2\sigma_0 &= \frac{2\sigma}{\sqrt{1+\frac{z\pi\theta^2}{\lambda}}} = \frac{\lambda}{\pi\theta}\label{eq: definition of beam waist} \\
    \tf z &= \sqrt{\frac{\pi\theta^2 2\sigma^2}{\lambda}-1}
\end{align}
allowing the inner of the second exponent on the RHS of \eqn{\ref{eq: gaussian pulse}} to be written:
 
\begin{equation}\label{eq: middle exponent of gaussian beam}
     \frac{k\sqrt{(x^2+y^2)}}{2R(z)} = 
    \frac{k\sqrt{x\2+y\2}}{2}\left(\frac{\pi\theta\22\sigma^2}{\lambda}+\frac{4\lambda\2}{\pi\2\theta^4}-1\right)^{-\frac{1}{2}} = \frac{k\sqrt{x\2+y\2}}{2\sqrt{\alpha}}
\end{equation}

\noindent allowing us to re-write \eqn{\ref{eq: gaussian pulse}}:
\begin{align}\label{eq: gaussian pulse solution}
\notag E(x,y,z = z, t) &=  \frac{E_0\lambda}{2\sigma\pi\theta}\exp\left[\frac{(x^2+y^2)}{2\sigma^2}\right]\\ 
&\times \exp\left[\frac{k\sqrt{x\2+y\2}}{2\sqrt{\alpha}}\right]\exp\left[ikz\right]\exp\left[\left(\frac{-2t}{\tau}\right)\2\right]
\end{align}

The total power of the field $E$\spatial is defined by the intensity:

\begin{equation}
    P_0 = \frac{1}{2}\pi I_0 2\sigma_0\2 = \frac{1\|E_0\|^2\lambda\2}{2\pi\theta\2}
\end{equation}

where the energy of the pulse is therefore:

\begin{equation}
    \notag U = \int_{-\tau/2}^{+\tau/2}P_0(t)= \frac{\tau^2\lambda\2}{\pi\theta\2}\|E_0\|
\end{equation}

Setting $U, \sigma,\theta = U_A, \sigma_A, \theta_A$, we define the analytical model of a pulsed Gaussian beam, $E_A$:

\begin{align}\label{eq: gaussian pulse solution}
\notag E_A(x,y,z = z, t) &=  \sqrt{\frac{\pi^2\theta\2U_A}{\tau\lambda\2}}\frac{\lambda}{2\sigma_A\pi\theta_A}\exp\left[\frac{(x^2+y^2)}{2\sigma_A^2}\right]\\ 
&\times \exp\left[\frac{k\sqrt{x\2+y\2}}{2\sqrt{\alpha_A}}\right]\exp\left[ikz\right]\exp\left[\left(\frac{-2t}{\tau}\right)\2\right]
\end{align}.

Analytical fits of the properties of the SASE1 undulator at the European synchrotron are given as a function of electron beam charge and photon energy in \cite{sinn_x-ray_2011} for photon energies in the range 5 keV - 40 keV. In pursuit of a geometric model of the SASE1 source, we assess the validity of \eqn{\ref{eq: gaussian pulse solution}} for beam properties defined by the analytical trends:

\begin{align}
    \label{eq: analytical pulse width} \sigma_A [\mu m] &=  a^{-1} 6\ln\left(\frac{6000}{\lambda[nm]} \right)\\ %%%% note need to change this to ratio w/ intensity
    \label{eq: analytical pulse divergence} \theta_A [\mu rad] &=  a^{-1} \left(13.9 - 5.17 Q[nC]\lambda[nm]^{0.85}\right)\\
    \label{eq: analytical pulse energy} U_A[mJ] &=  15.4 Q[nC]\lambda[nm]\\
    \label{eq: analytical pulse duration} \tau_A[fs] &= \frac{10^3 Q[nC]}{9.8}
\end{align}

\noindent where Q is the electron beam bunch charge, square brackets denote units and $a = \frac{1}{\sqrt{2ln(2)}}$ is a conversion factor between the Gaussian beam width and full-width-half-maximum (FWHM): $\sigma = \frac{FWHM}{a}$.\\

We define the energy of an arbitrary pulse, $E\spacetime$ to be the sum of transverse intensities intersecting some longitudinal plane, z, integrated over the duration of the pulse:

\begin{equation}\label{eq: definition of pulse energy}
    U(z) = \iint E(x,y,z,t) dxdydt.
\end{equation}

For calculation of the beam width and divergence, we choose to define the encircled energy of a beam in spherical coordinates:

\begin{equation}\label{eq: encircled energy}
    E(r) = \frac{\int_0^r I(r') 2\pi r' dr'}{\int_0^\infty I(r') 2\pi r' dr'},
\end{equation}
\noindent and solve for r:

\begin{equation}\label{eq: enclosed energy algorithm}
    \argmin_{x\in\mathbb{R}} \left\|\frac{\int_0^r I(r') 2\pi r' dr'}{\int_0^\infty I(r') 2\pi r' dr'} - 0.5 \right\|
\end{equation}
so that the beam size is defined by a radius enclosing 50\% of the total beam intensity, centred at the center-of-mass of the beam, $\boldsymbol{r}_0$:

\begin{equation}\label{eq: beam com}
    \boldsymbol{r}_0 = \frac{1}{\int I_z(\boldsymbol{r})d\boldsymbol{r}}\int I(\boldsymbol{r})\boldsymbol{r} d\boldsymbol{r}.
\end{equation}

\noindent We note that this can be related to the width of a Gaussian beam which has an encircled energy, U(r):

\begin{equation}\label{eq: encircled energy of a gaussian}
    U(r) = 1-\exp\left[-2\left(\frac{r}{2\sigma}\right)\2\right]
\end{equation}
setting $E(r) = 0.5$

\begin{align}
\notag    \frac{r}{2\sigma} &= \sqrt{\frac{\log(0.5)}{-2}} \approx 0.588 \\
\tf 2\sigma &= \frac{r}{\sqrt{\frac{\log(0.5)}{-2}}}.
\end{align}

Finally, we define the beam divergence from its FWHM of the electric field in k-space:

\begin{align}
    \notag \hat{E}(\kx,\ky) &= \frac{1}{2\pi}\iint E(x,y)\exp\left[-i\left(\kx x + \ky y\right)\right] dxdy \\
    &= \ifourier\left[\hat{E}(\kx,\ky)\right]
\end{align}

where we solve for $\theta = a^{-1}k_\perp$:

\begin{equation}
    \argmin_{k_\perp\in\mathbb{R}} \left\|\frac{\hat{E}(0)}{\hat{E}(k_\perp)}-\frac{1}{2}  \right\|. 
\end{equation}

Figure \ref{f: coherent source properties} compares the modelled properties of the beam with analytical expectations and demonstrates the suitably of \eqn{\ref{eq: gaussian pulse solution}} in representing the analytical properties of the SASE1 undulator at the European XFEL in the energy range 5 keV - 40 keV.

\begin{figure}[H]
\centering
\includegraphics[width=1\textwidth]{figures/coherent_source_properties.png}
\caption{Comparison of the pulsed Gaussian beam model described in \eqn{\ref{eq: gaussian pulse solution}} and the prescribed analytical beam a) divergence, b) pulse energy, c) pulse duration and d) 2$\sigma$ pulse-width. Error bars in d) correspond to the error associated with implementation of \eqn{\ref{eq: enclosed energy algorithm}}}
\label{f: coherent source properties}
\end{figure}


\subsection{Modelling Beamline Optics}
\subsubsection{Description and Validation of the Mirror Materials}
\subsubsection{Description and Validation of the Mirror Apertures}
\subsubsection{Description and Validation of the Mirror Surfaces}

\subsection{The Coherent Focus}


\subsection{The Partially Coherent Source Model}
\subsubsection{Description of the Partially Coherent Source}
\subsubsection{Comparison to Experiment}




\section{Results and Discussion}
\subsection{Properties of the Partially Coherent Focus}
\subsection{Discussion}

\section{Conclusions}
