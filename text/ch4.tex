
\section{Introduction}
XFEL source instability is a fundamental inhibitor to the characterisation of the structure and dynamics of single proteins and macromolecule assemblies  \cite{neutze_opportunities_2014, oberthur_biological_2018}. While the high brilliance and ultra-fast repitition rate of XFEL sources has enabled the developement and advancement of Coherent Diffractive Imaging (CDI) of non-crystalline biological samples, Serial Femtosecond Crystallography (SFX) and MegaHertz (MHz) microscopy \cite{,vagovic_megahertz_2019, ekeberg_single-shot_2016, chapman_femtosecond_2006, pandey_time-resolved_2020, gunther_sequential_2011} the inherent stochastic properties of the SASE x-ray generation process limit the degree to which the potential of these techniques can be realised \cite{spence_x-ray_2012, sun_current_2018, aquila_linac_2015}. While the detriment of variances in pulse fluence, spatial energy density, source coherence and beam pointing pervade XFEL literature  \cite{nakano_single-particle_2018, nagaya_ultrafast_2016, hantke_high-throughput_2014, loh_cryptotomography_2010} the origin of these instabilities is poorly understood in the absence of suitable wavefront characterisation methods. Preliminary whitefield data (see below) collected using the SPB-SFX instrument at the European XFEL suggests that electron-beam parameters play a role in the pulse-pulse and train-train correlations of the source. In pursuit improved source stability, we propose a novel, single-shot MHz phase-retrieval mechanism suitable for demonstrating the role of pulse duration, bunch length, electron beam charge and energy on the coherence of the focal-plane wavefield.

\section{Virtual Reference Single-Shot Phase-Retrieval}
Efforts to understand and remediate phase aberrations in optical sources are pervaded by the optical-phase problem: no current detector is capable of imaging the phase and amplitude of an optical wavefront simultaneously. While phase-retrieval mechanisms for synchrotron and visible light sources are well-developed, the pulse-to-pulse instability of XFEL sources are poorly represented in classical \cite{sala_ptychographic_2017} and speckle-based \cite{morgan_ptychographic_2020, berujon_x-ray_2015} multi-shot ptychographic wavefield reconstructions. While methods for extracting independent pulses from ptychographic reconstructions of the ensemble beam phase have been reported \cite{sala_pulse--pulse_2019}, the computational requirements of iterative phase-retrieval techniques likely prohibit the development of \textit{online} optical phase measurement and correction infrastructures. Fine-structure variations throughout XFEL trains present a parallel challenge to dedicated single-shot grating and interference based techniques that have otherwise seen success in widefield sensing for astronomy \cite{potier_comparing_2020} and  adaptive optics \cite{abraham_development_2016}, including but not limited to multi-slit diffraction \cite{lupini_aberration_2010,paterson_spatial_2001, wachulak_spatial_2017}, Shack-Hartmann Sensors \cite{freisem_spectrally_2018, keitel_hartmann_2016} and speckle-contrast analysis \cite{yun_coherence_2019, zanette_speckle-based_2014}, due to the imposition of stringent fabrication requirements to adequately sample phase variations at or near the focus of the XFEL beams \cite{soldevila_phase_2018}.\\

We present a correlation-based wavefield characterisation method derivative of single-shot phase contrast imaging techniques \cite{paganin_tutorials_2019} equivalent to scanning speckle-based wavefront characterisation techniques \cite{zdora_state_2018} as well as geometric solutions to wavefront phase using the transport of intensity equation \cite{gureyev_phase_1995, zuo_boundary-artifact-free_2014, pavlov_single-shot_2019}. Importantly, the following technique is free from the requirement of a reference image, makes no assumption on the uniformity of the incident pulse-intensities and is achievable using readily available materials.\\

Within the scope of optical wavefront sensing and phase retrieval, we re-imagine the sample thickness determination technique developed in \cite{morgan_quantitative_2011} pulse-to-pulse perturbations in an XFEL light source. Hence, we approach a problem similar to (oriental magic mirrors, berry)\\

\subsection{Absolute Mode}
Consider a uniform, random phase-mask, defined by the optical transfer function S(x,y):
\begin{equation}
    S(x,y) = exp(i\phi(x,y)) = exp(ik\delta(x,y)T(x,y)),
\end{equation}
where $\delta(x,y)$ is the real component of the refractive index and T(x,y) is thickness of the mask at each point in the transverse plane.\\ 

Suppose the mask was placed in the plane immediately prior to the detector, then a phase contrast reference image, $I_R(x,y)$ would be recorded:

\begin{equation}
    I(x,y) = \|S(x,y) \|^2,
\end{equation}
where each feature in the mask is unperturbed and hence, the phase gradient at each point in the detected image is $\partial\phi_{x,y} /\partial (x,y) = \theta_{x,y} = 0$. The recorded image is labelled the virtual reference.\\

\textbf{Note:} The virtual reference satisfies the condition that:

\begin{equation}
    S(x,y) = \mathcal{F}^{-1} \left[i\Delta z (\sqrt{k^2-k^2_x-k^2_y)} ) \right]\mathcal{F} \times S(x,y).
\end{equation}

With the reference wavefield in hand, we are capable of comparing the phase perturbations present in the beam by mapping the transverse shift of phase features embedded in the beam, over the distance $\Delta z$. Consider a pulsed wavefield $\psi_n$ with characteristic amplitude and phase distributions $A_n(x,y)$ and $\phi_n(x,y)$:
\begin{equation}
    \phi_n(x,y) = A_n(x,y) exp[-i\phi_n(x,y)]
\end{equation}

\begin{equation}
   I_n(x,y) = \| \mathcal{F}^{-1} \left[i\Delta z (\sqrt{k^2-k^2_x-k^2_y)} ) \right]\mathcal{F} \times \psi_n(x,y) S(x,y) \|^2.
\end{equation},
where we pursuit an absolute solution for $\phi_n$.

By comparison of the unperturbed reference field with a nominal phase distribution, we can calculate the phase of the perturbed wavefield by measurement of the transverse shift of features in the array. A feature originating at some point, $(x_i,y_i)$ in the interaction plane, that is shifted by $\delta x$ and $\delta y$ in each transverse dimension gives a phase gradient at the feature origin:
\begin{equation}
    tan(\theta_{x,y}) = \frac{\delta x/ \delta y}{\Delta z},
\end{equation}
where 
\begin{equation}
    \theta_{x,y} = \frac{1}{k}\frac{\partial \phi}{\partial x / \partial y}.
\end{equation}
Making use of the Fourier derivative theorem:
\begin{equation}
    \mathcal{F}\frac{\partial \psi}{\partial r} = ik_r \mathcal{F}\psi,
\end{equation}
a solution for the phase distribution of the pulse is given:
\begin{equation}
    \phi_n(x,y) = \mathcal{F}^{-1}\left[\frac{\mathcal{F}\left(\frac{\partial \phi_x}{\partial x} + i\frac{\partial \phi_y}{\partial y}\right)}{ik_x-k_y}\right].
\end{equation}
where the origin of Fourier space is set: $\phi = 0$.

\subsection{Special Case}
We reserve room for the special case that the beam curvature is sufficiently large in comparison with the beam gradient that the virtual reference of the detector plane image must be scaled accordingly (see Fresnel scaling theorem).


\subsection{Differential Mode}
In the absence of a well-defined reference structure, the phase differential between any set of pulses can be interpreted directly from a pair of intensity images. The relative shift of phase features in the detector plane is a depiction of variations in pulse fluence and beam pointing and can consequently be seen as a natural descriptor of the coherence of a wavefield.\\

For a set of consecutive pulses illuminating a phase object, $S$, the difference in phase gradient is given by the transverse shift of features in the detector plane:

\begin{equation}
    tan(\Delta \theta_{x,y}) = \frac{\delta x/ \delta y}{\Delta z},
\end{equation}
where 
\begin{equation}
    \Delta \theta_{x,y} = \frac{1}{k}\frac{\partial \phi}{\partial x / \partial y}.
\end{equation}
